\documentclass[12pt,letterpaper]{article}
\usepackage{fullpage}
\usepackage[top=2cm, bottom=4.5cm, left=2.5cm, right=2.5cm]{geometry}
\usepackage{amsmath,amsthm,amsfonts,amssymb,amscd}
\usepackage{lastpage}
\usepackage{enumerate}
\usepackage{fancyhdr}
\usepackage{mathrsfs}
\usepackage{xcolor}
\usepackage{graphicx}
\usepackage{listings}
\usepackage{hyperref}
\usepackage{ragged2e}
\usepackage[version=4]{mhchem}
\usepackage[shortlabels]{enumitem}


\hypersetup{%
  colorlinks=true,
  linkcolor=blue,
  linkbordercolor={0 0 1}
}
 
\renewcommand\lstlistingname{Algorithm}
\renewcommand\lstlistlistingname{Algorithms}
\def\lstlistingautorefname{Alg.}

\lstdefinestyle{Python}{
    language        = Python,
    frame           = lines, 
    basicstyle      = \footnotesize,
    keywordstyle    = \color{blue},
    stringstyle     = \color{green},
    commentstyle    = \color{red}\ttfamily
}

\setlength{\parindent}{0.0in}
\setlength{\parskip}{0.05in}


% Edit these as appropriate
\newcommand\course{MAT 214}
\newcommand\hwnumber{1}                  % <-- homework number
% \newcommand\NetIDa{netid19823}           % <-- NetID of person #1
% \newcommand\NetIDb{netid12038}           % <-- NetID of person #2 (Comment this line out for problem sets)

\pagestyle{fancyplain}
\headheight 35pt
% \lhead{\NetIDa}
% \lhead{\NetIDa\\\NetIDb}                 % <-- Comment this line out for problem sets (make sure you are person #1)
\chead{\textbf{\Large Homework \hwnumber}}
\rhead{\course \\ \today}
\lhead{Due Feb. 14 @ 7:55 am}
\lfoot{}
\cfoot{}
\rfoot{\small\thepage}
\headsep 1.5em

\begin{document}

\section*{Problem 1 - 16 points}
Which of the following materials can be classified as a ceramic? Provide justification. If the material is a ceramic please provide one example application where the materials is used.

\begin{enumerate}[(a)]
    \item Solid Argon
    \item Molybdenum diselenide (\ce{MoSe2})
    \item Solid Salt \ce{NaCl}
    \item Ice
    \item Boron Carbide \ce{B4C}
    \item Polyethylene
\end{enumerate}

\textbf{Points Breakdown:}\\[12pt]
2 points: Identify correct ceramic material or not\\
1 point: Identifying application where ceramic material is used.\\
Note, do not use the points as an indicator of how many ceramic materials are included. You will be misled.\\[6pt]
\textbf{Total points: 16}

\section*{Problem 2 - 16 points} 
The following definition of ceramic material were proposed. Explain if they are good definitions or not.

\begin{enumerate}[(a)]
    \item "All ceramics are opaque"
    \item "All ceramics have covalent, ionic, or mixed ionic covalent bonds"
    \item "All ceramics are good insulators"
    \item "All ceramics are brittle"
    \item "All ceramics are solids"
\end{enumerate}

\textbf{Points Breakdown:}\\[12pt]
1 point for a general statement on if these are good definitions. Are they necessary conditions or definitions?\\
3 points for each correct justification.\\[6pt]
\textbf{Total points: 16}

\section*{Problem 3 - 16 points}
Tiles used for spacecraft reentry are made of silica yet have a low thermal conductivity of ~0.028 $BTU/ft-hr-^{\circ}F$. Fused silica, a chemically identical material has a thermal conductivity of ~9.6 $BTU/ft-hr-^{\circ}F$. Why is there an orders-of-magnitude difference in the thermal conductivity. 

\textbf{Total: 16  Points}

\section*{Problem 4 - 16 points}

Describe \textbf{two} reasons why you might want to design a pore free ceramic material to achieve a property for a specific application. Make sure you mention both the required material property and the application. \\

\textbf{Points Breakdown:}\\[12pt]
4 points for each property described\\
4 points for each application described\\[6 pt]
\textbf{Total: 16 Points}

\section*{Problem 5 - 16 points}

What is a refractory ceramic? What applications are they used for? Describe at least 3 essential properties of refractory ceramic materials?\\

\textbf{Points Breakdown:}\\[12pt]
2 points for definition of a refractory ceramic.\\
2 points for applications where refractory ceramics are used. \\
4 points for each property essential for refractory ceramics. \\[6 pt]
\textbf{Total: 16 points}


\section*{Problem 6 - 16 points}

Draw a qualitative plot indicating the thermal evolution of a concrete setting reaction over time. Make sure to indicate the time scales of the reaction. Explain why thermal management is important when considering concrete? Which part of the reaction is most important to consider and why?\\

\textbf{Points Breakdown:}\\[12pt]
8 points for graph\\
2 points for correct reaction time scales\\
3 points for why thermal management is important\\
3 points for which part of the reaction is most important to consider\\[6 pt]
\textbf{Total: 16 Points}

\end{document}
